\documentclass[11pt,a4paper]{article}
\usepackage[utf8]{inputenc}
\usepackage[T1]{fontenc}
\usepackage{lmodern}
\usepackage[margin=1in]{geometry}
\usepackage{graphicx}
\usepackage{hyperref}
\usepackage{booktabs}
\usepackage{amsmath}
\usepackage{amssymb}

\title{Brief Overview of Sumerian Grammar}
\author{AncientGrok Research}
\date{January 17, 2026}

\begin{document}

\maketitle

\documentclass{article}
\usepackage[utf8]{inputenc}
\usepackage{amsmath}
\usepackage{geometry}
\geometry{margin=1in}
\title{Brief Overview of Sumerian Grammar}
\author{AncientGrok Research}
\date{\today}

\begin{document}

\maketitle

\section{Introduction}
Sumerian, one of the world's oldest written languages, was spoken in southern Mesopotamia from ca. 3100--2000 BC. It is an isolate language, unrelated to any known family, attested primarily through cuneiform texts. Sumerian grammar is agglutinative and highly inflectional, characterized by extensive use of affixes for grammatical relations. This report provides a concise overview of key features, drawing from standard references such as the Electronic Text Corpus of Sumerian Literature (ETCSL) and grammars by Thomsen (1984) and Woods (2008).

\section{Phonology and Orthography}
Sumerian had approximately 16 consonants and 5 vowels (/a, e, i, u, \'e/ where \'e is a high central vowel). Cuneiform script is logo-syllabic: signs represent words, syllables, or determinatives. Sumerian is often analyzed with a consonant-vowel-consonant (CVC) syllable structure. Key phonological features include vowel harmony in some suffixes and gemination avoidance.

\section{Nominal Morphology}
Nouns are inflected for number, case, and class via prefixes and suffixes:
\begin{itemize}
\item \textbf{Number}: Singular (default), plural (-ne), collective (-\'a{k}).
\item \textbf{Case system}: Ergative-absolutive alignment. 9--12 cases including:
  \begin{itemize}
  \item Absolutive (unmarked, subject of intransitive, object of transitive).
  \item Ergative (-e), Genitive (-ak), Dative (-\'a{k}), Locative (-a), Ablative (-ta).
  \end{itemize}
\item \textbf{Dimensional prefixes}: Prepositional elements like \textit{ta-} `from', \textit{ši-} `to'.
\end{itemize}
Example: \textit{lugal} `king' → \textit{lugal-e} `by the king (ergative)'.

\section{Verbal Morphology}
Verbs are complex, with a prefix chain (up to 9 slots) + root + suffixes:
\begin{enumerate}
\item Prepositional prefixes (e.g., \textit{i-} `eye/see').
\item Dimensional prefixes (location/motion).
\item Conjunctional prefixes (e.g., \textit{u-} `and').
\item Agent prefix (\textit{ŋa-} `I', \textit{e-} `he/she').
\item Dimensional infix.
\item Voice prefix (\textit{ba-} middle/passive, \textit{mu-} causative).
\item Main chain ends with root + tense/aspect suffix.
\end{enumerate}
Tenses: Present-future (-ø/-\'en), Past (-ed).
Example: \textit{ŋa-mu-un-du₃} `I caused it to be built'.

\section{Syntax}
\begin{itemize}
\item \textbf{Word order}: SOV (Subject-Object-Verb), postpositions.
\item \textbf{Ergativity}: Intransitive subject and transitive object pattern together (absolutive); transitive subject is ergative.
\item \textbf{Subordination}: Relative clauses via nominalization (verbal noun + possessor).
\item \textbf{Negation}: Prefix \textit{nu-} or \textit{na-}.
\end{itemize}

\section{Key References}
\begin{itemize}
\item Thomsen, Marie-Louise. \textit{The Sumerian Language}. Akademisk Forlag, 1984.
\item Woods, Christopher. \textit{Sumerian Grammar}. Johns Hopkins University, 2008 (lecture notes).
\item ETCSL: \url{http://etcsl.orinst.ox.ac.uk/} for texts and grammar sketches.
\end{itemize}

\section{Conclusion}
Sumerian grammar's agglutinative nature and ergative system distinguish it from Semitic Akkadian, which largely replaced it. Ongoing research refines reconstructions via comparative philology.

\end{document}


\end{document}
